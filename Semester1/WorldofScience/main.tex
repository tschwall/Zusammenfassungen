\documentclass{article}
\usepackage[utf8]{inputenc}
\title{English_V1.0}
\author{Thierry Schwaller}
\date{January 2020}
\usepackage[a4paper,width=150mm,top=25mm,bottom=25mm]{geometry}
\pagestyle{myheadings}
\markright{Thierry Schwaller}
\begin{document}
\section{Writing scientific papers}
\subsection{Important words}
Plagarism $\rightarrow$ Taking ideas or words from a source without giving credit
\subsection{Summary writing}
Should be about a third of the original text. Try to note down the keywords and give the basic idea of the text. There should be no personal opinions.
\subsection{Reports}
Reports are used as a way of communication in the workplace. It should be an objective organized presentation of factual information to answer requets. There are three different types: observation reports (results of an experiment), progress report (status of a project) and the feasibility report (analysis of the possibilities and presentation of the recommendations).
\subsection{Academic writing}
Different types: Notes / Reports / Essays / Papers / Disserations \\
Different parts: 
\begin{itemize}
    \item Abstract
    \subitem Background $\rightarrow$ Present
    \subitem Purpose of study $\rightarrow$ Present / Past
    \subitem Methodes used $\rightarrow$ Past
    \subitem Results $\rightarrow$ Past
    \subitem Conclusion $\rightarrow$ Present / Future
    \item Introduction
    \subitem Which field and why
    \subitem Claiming a gap
    \subitem How do you intend to fill this gap
    \item Methods
    \subitem What was done how
    \subitem Justify your methods
    \subitem Describe material / how was it prepared / describe research protocols / measurements
    \item Results
    \subitem Preparatory (Vorbereitende) information
    \subitem Reporting results
    \subitem Commenting on the results
    \item Conclusion and discussion
    \subitem Recommendations for the future
    \subitem Meaning of the study
    \subitem Review of the findings
\end{itemize}
\newpage
\section{TED - Talks}
\subsection{Champion of american science (Joseph Henry)}
He worked at the smithsonian institute. He was a great teacher and got mostly famouse for his physics lectures and experiments. He also created one of the strongest electromagnets in his time.
\subsection{The tradeoffs of building green (Catherine Morh)}
She talks about embodied energy (graue Energie) for example how much water or energy is used for creating a paper towel or washing a sponge and how you can improve your eco friendliness while building a house.
\subsection{The surpringsing way ISIS stay in power (Bernedette Bertis)}
She begins explaining how war has changed from a fight between states to a fight between states and non state organistation (like ISIS).\\
Non state terror organisations often use something that the government isnt providing and lures the people with that into the organisation, like education, saftey or public health. To stop them the government needs to fill these gaps in service. 
\subsection{How language transformed humanity (Mark Pagel)}
He talks about how language is one of the most dangerous and effective weapons. He mentions that the key difference between humans and animals is social learning, or as he also calls it visual theft, but this can be dangerous because someone could steel our ideas thats why we invented speech and languages to communicate.
\subsection{How algorithms shape our world (Kevin Slavin)}
He talks about algorithms and of how we created something that we cant fully understand. Algorithms are used in our everyday live from Netflix (movie recommendation) to ads that pop up on our computer
\subsection{Building a park in the sky (Robert Hammond)}
RObert talks about how he created a park in downtown new york from an old high line for trains.
\subsection{Lets use videos to reinvent education (Salman Kahn)}
Salman is the owner of a youtube channel named Kahn academy where he publishes learining videos. The advantages of video learnings according to him are you can study at your own pace, you can do it in your own living room, you dont have to be embarrassed if you dont know something and you can collect data about yourself which helps you find your weak spots. \\
It can also be used in schools so the teacher have more time to talk to the pupils individually (as he calls it "humanizing the class room").
\subsection{How not to be ignorant about the world (Rosling)}
People often think the world is way worse than it acctually is. Our preconceptions about a lot of topics have to change. \\
There are three main areas which influence our preconceptions: personal bias, outdated information and news bias. 
\subsection{The danger of a single story (Ngozi)}
She talks about how seeing only one view can influence our thinking and changed the way we look at people. For example our preconceptions about african people. She also mentions how starting a story at different points can drasticly change our thoughts.
\subsection{Let my dataset change your mind (Hans Rosling)}
We should stop dividing our world into the wester world and the developing world, this thought is outdated (view of the students correlates to the year the teacher was born). We need to look at each country individually.
\subsection{The magic of washing machine (Hans Rosling)}
We need to change our energy consumptions because more and more people want washing machines (want an easier life).
\subsection{Religion and babies(Hans Rosling)}
Religion and child birth rates are often put into correlation to each other, but as Hans Rosling puts it this is an outdated view of ther world. Today there is no clear correlation between child birth and which religion the parents are. The correlations is more between the child mortality rate (higher child mortality the more children are born).
\subsection{Where joy hides and howto find it (Ingrid Fetell Lee)}
She talks about what joy is and how it is connected to the physical world. She found out that people from different ages and different nations feel joy in a very similar way. Curves make us feel safe in comparission to edges. Colors give us a feeling of home and life.
\subsection{The magic science of storytelling (David J Phillips)}
Stories are something extremly powerfull. They can make us buy or think things (product placement). Stories mainly release three positive chemicals in our brain, the same as falling in love, Endorphins, oxytocin and dopamin. 
\subsection{3 principles for creating safer AI (Stuart Russell)}
The recent disvoceries will change our world and it it one of the biggest discoveries in human history. The problem it that we could create something more intelligent than us. The "midas problem" is stating something that we dont really want.
\newpage
\section{E=m$c^2$}
\subsection{Bern Patent Office}
The first chapter talks about the life of Einstein before he got famous. After graduating from the ETH he took a job at a patent office. He was not really successfull with his career until he published his first article in "Die Analen der Physik" in 1905. This was his turning point.
\subsection{E is for energy}
\textbf{Michael Faraday}\\
He started as a bookbinder with almost no formal education but this helped him to have no preconseptions about how electricity and magnetism work together. \\
His most important discovery was the \textbf{Law of the conservation of Energy}\\

\textbf{Humphrey Davy} \\
Was the mentor of Faraday but he claimed that he has discovered the Law of the consercation of energy.
\subsection{=}
Talks about how the equal sign got invented. Most likely invented by an english man by the name of Robert Recorde.
\subsection{m is for mass}
\textbf{Lavoisier} \\
Lavoisier was a french physicist who discovered the \textbf{Law of the conservation of mass}. (That mass can never be destroyed or created) He found this while looking at the rusting process of iron which would way more after it had rusted, because it sucked some of the gasses up. \\
He was later killed in the french revolution because he helped the king to build a wall around paris, this alone wouldnt have been a problem wasnt there \textbf{Jean-Pail Marat} a man who lived in poverty because of Lavoisiers rejection of him. \\
\\
Einstein could later \textbf{link the conservation of Energy and the conservation of mass} .
\subsection{c is for celeritas}
The first person to try and measure the velocity of light was Galileo but he came to the conclusion that light travels infinitely fast.\\
\textbf{Ole Roemer} \\
While analysing the movement of one of the Jupiter moons he discovered that there was a difference in the movement. He discovered that it was because of the earth being closer or further away from it and so he concluded that the light had to travel at a finite speed. 
\textbf{Cassini}\\
He was a more well known physicist than Roemer at the time of Roemers discoveries and because they contraticted Cassinis statements he didnt accept it that he was wrong. And because of that europeans astronomers still did not accept that light traveled at a finite speed. \\
\textbf{Einstein} \\
Einstein discovered the closer you get to the speed of light the heavier an object gets.
\subsection{$^2$}
\textbf{Emilie du Chatelet}
She was a french physicist from a rich home. She proved that $E = mv^2$ by putting together different experiments and papers. (Prooving that Leibniz was right and not Newton).
\subsection{Einstein and the equation}
\textbf{Marie Curie}\\
Marie Curie was a polish physicist is known for her work on radioactivity (Radium / Polonium), the  before dying from radiation. \\
\textbf{Einstein}\\
The ground-breaking this about Einsteins equation what that he could connect the separeted law of conservation of energy and the law of conservation of mass with such a simple formula.\\
Einstein never liked the term "relativity" because it gave a sense of no exact results, which was not the case the predictions where precise. 
\subsection{Into the atom}
\textbf{Ernest Rutherford}\\
He discovered that an atom is almost enirely made out of free space with a small concentrated center and of a few electrons flying around.\\
He also disvoered that the energy Einstein discovered in his equation has to be hidden in the center because it could keep the extremly fast electrons in a near distance to the center.\\
\textbf{James Chadwick}\\
James Chadwick was the assistent of Rutherford and he discovered a new part in the center of the atom the neutron.
\subsection{Quiet in the Midday Snow}
\textbf{Otto Hahn and Lise Meitner}
They worked together in a laboratory but because of Lise Meitners jewish background she had to flee from germany to sweden. Where she then discovered with the experiments of otto hahn the nuclear fission. (How to get the energy which einstein spoke about out of the atom)
\subsection{Germanys turn}
Einstein wrote a letter to the president of the united states to inform the president about the possibilities of uranium as a energy and powerful weapon source. He was ignored.\\
\textbf{Werner Heisenberg} \\
Heisenberg was a german physicist who was at first not accepted by the SS because another physicist told them that he was not patriotic enough. \\
But because the mother of Heisenberg was a friend of the mother of Heinrich Himmler Heisenberg was accepted and started working on the atomic bomb. His bomb design was based on Fermis idea (slwoing the neutrons down to increase the nucllear reaction) but instead of water he used heavy water which would slow down the neutrons even more. \\
The biggest error the german did with there bomb is that they used flat sheets of uranium and not spheres as the US.
\subsection{Norway}
To slow down the progress of the germans england attacked and destroyed the only factory in which heavy water was produced in norway.
\subsection{Americas Turn}
\textbf{Oppenheimer}\\
Oppenheimer was the head of the \textbf{Manhatten Project}. He had different teams working on different ideas, but the most effectiv idea was to shoot with a gun loaded with plutonium at plutonium.\\
Oppenheimer was later monitored by the US government because he spoke in public that the usage of the bomb was not justified. Heisenberg returned being a professor after the war.
\subsection{8.16 A.M - Over Japan}
On the 6. August, 1945 at 8.16 A.M. the US released the first atomic bomb over hiroshima. The chain reaction started at around 1900 ft to destroy as much as possible.
\subsection{The Fires of the sun}
\textbf{Cecilia Payne}\\
Payne reinvented the spectroscope (a measurement device which breaks up the spectrum of the electromagnetic radiation into various wavelengths allowing them to be identified). \\
With the help of her new spectroscope she discovered that the sun is almost fully made up out of hydrogen. The energy the sun releases comes from the hydrogen atoms fuse to one helium atom which realeases a lot of energy in the process.
\subsection{Creating the earth}
\textbf{Hoyle} \\
Hoyle duscivered that an imploding start could reach close to 100 million degrees, which is enough energy to squeeze large nulcei of more massive elements together and form new elements. He got this idea from analysing the bomb droped on hiroshima. \\
\textbf{Modern application of E=mc2} \\
Smoke detectors, exit signes, PET scans and archaeology.
\subsection{A Brahmin lifts his eyes unto the sky}
The sun will run out of hydrogen in about 5 billion years, which will cool the earth down and because of the lose of the gravitational pull the earth will fly away\\
\textbf{Chandra} \\
He came up with the concept of black holes

\end{document}
